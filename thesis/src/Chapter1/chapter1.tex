\chapter{Introduction}
\graphicspath{{Chapter1/Figs/}{Chapter1/Figs/}}

This chapter introduces the reader to the primary focus, key topics, and broad explanations of this thesis. It also presents the supposition, goals, and objectives of its primary content and structure.

\section{Background}
\label{chapter1-background}

There has been a long-standing interest in developing neural interfaces—systems that sense and interact with the nervous system’s electrical activity. Successful research into the development of technologies that enable neural interfacing has been underway for decades, with the first experiments being conducted by Jacques J. Vidal in the late 1970s \citep{vidal_real-time_1977}. In particular, a related discipline focusing on the direct interaction between brains and computers via a brain–computer interface (BCI) has gained momentum with the emergence of companies such as Neuralink \citep{mor_brain-computer_2021}.

One aspect of BCIs is the development of imaging technologies that enable the measurement of brain activity. A distinction can be made between different methods of measuring brain activity signals at different locations. On the one hand, there are invasive sensors used with electrocorticography (ECoG), a sensor-based\footnote{Other options next to sensor-based imaging methods are, for example, scan-based methods such as computed tomography (CAT) that utilises special X-rays to produce axial images of the brain.} imaging method that places electrodes on the surface of the brain; on the other hand, non-invasive sensors are placed on the body, such as with electroencephalography (EEG). Both methods measure the electrical field elicited by the firing of neuronal populations. However, the further away the electrode is from the brain and the more tissue (e.g. scalp, skull, cerebrospinal fluid, and cortex) lies between firing neurons and the measurement sensor, the more the spatial resolution decreases.

A second aspect of BCIs is the development of software that reads and interprets data from sensors. Both aspects present their own set of challenges and complexities. Nonetheless, complete and applicable BCIs work in practice and have been used for many years in patients with neurological disorders \citep{braingate_publications_nodate}. There are also consumer and non-clinical BCIs available, such as the OpenBCI and Neurosity products, which aim to democratise the use of EEG systems by offering low-cost hardware and open-source software.

\section{Relevance}
\label{chapter1-relevance}

The possibilities for sufficiently and directly connecting the human brain to the outside world via computers are seemingly endless, given the purely physical\footnote{There is research on the quantum mind that describes how classical mechanics cannot explain consciousness; however, this is not considered in this thesis as this research is still in its early stages.} assertion that all our feelings, memories, dreams, and thoughts are most likely the sum of electrical activities in our brain. There are several use cases for utilising insights from our brain to interface with computers, such as controlling prosthetic limbs for amputees \citep{campbell_amputee_2014} as shown in \autoref{fig:prostetic-arms}, enabling communication for people with locked-in syndrome\footnote{Locked-in syndrome describes the paralysis of all voluntary muscles in its entirety, thereby making it impossible for people to communicate with the outside world.} \citep{chaudhary_spelling_2022}, or diagnosing neurological problems and improving the mental capacities of elderly patients \citep{belkacem_brain_2020}. These are just a few of the promising examples that could be cited.

\begin{figure}[ht]
  \centering
  \includegraphics[width=\linewidth]{prostetic-arms.jpg}
  \caption[An amputee is using his mind to control two robotic arms to perform several tasks that require fine motor control.]{An amputee is using his mind to control two robotic arms to perform several tasks that require fine motor control \citep{campbell_amputee_2014}.}
  \label{fig:prostetic-arms}
\end{figure}

These examples make it evident that BCIs can significantly impact the field of therapeutics and accessibility for a small subset of the human population. However, one can envision not only alleviating deficient living conditions but also improving the lives of healthy people through more natural or efficient ways of interacting with technology or by directly altering human brains for certain benefits such as the possibility to enhance or delete bad memories \citep{spiers_enhance_2014} or record and guide dreams \citep{haar_horowitz_dormio_2020}.

Many use cases still seem a long way from being applicable today, yet many experts and even entire companies are developing BCI hardware and software, such as Neuralink, that is aimed at the general population \citep{urban_neuralink_2017}. The general applicability of a BCI system to the mass market will depend on several factors, of which the form factor and invasiveness of the hardware are likely to be essential aspects. Nevertheless, the totality of the ecosystem in which the software resides is a valuable aspect that should not be overlooked.

\section{Opportunity}
\label{chapter1-opportunity}

Whether it is a bidirectional and invasive BCI or a unidirectional and non-invasive BCI, the data collected from the brain would always need to be processed, contextualised, and classified to produce an intelligible output to interface with, as shown in \autoref{fig:directional-system}.

\begin{figure}[ht]
  \centering
  \includegraphics[width=\linewidth]{directional-system.png}
  \caption{Conceptual difference between a unidirectional and a bidirectional BCI and a simplified overview.}
  \label{fig:directional-system}
\end{figure}

Most current BCI software systems being developed—for example, for a BCI implanted in a living patient—are typically deployed in a local environment; that is the software system and its components are located on a physically nearby computer.

The author sees an opportunity to move BCI software from local environments to the cloud to enable a variety of benefits for the general population and mass market. The present thesis determines what components would be required by a cloud-based software system that is ready for the mass market. The emphasis is on a holistic view of such a system, which means that the entire technology stack and context are taken into account.

\section{Supposition}
\label{chapter1-supposition}

There is already promising research on the implications of brain/cloud interfaces (B/CI) by \citeauthor{martins_human_2019} \parencite*{martins_human_2019} and \citeauthor{angelica_cognitive_2021} \parencite*{angelica_cognitive_2021}, which analyses bringing hypothetical large-scale BCI software systems into the cloud. Nonetheless, their research focuses on hypothetical scenarios in the future, usually premised on the development of other technologies such as neural nanorobotics, vital advances in 5G, or the presence of supercomputers in the cloud (e.g. for the augmentation of the human brain); thus, they are somewhat removed from today’s realities. To distinguish the research presented in this thesis, the author coins the term ‘neural/cloud interface’ (N/CI), which refers to a holistic software system that connects a neural interface device such as a BCI to the cloud and then to other neural interfaces, software systems, or physical devices.

The primary supposition is that an N/CI is feasible with contemporary software technologies, requiring only theoretical groundwork based on empirical software engineering. To shed more light on this, this thesis looks at the process and lessons learned from the author’s perspective in developing a real-life N/CI in the industry.

\section{Goals and objectives}
\label{chapter1-goals-and-objectives}

The overarching goals of this thesis are to provide an overview of the context and definition of an N/CI and, most importantly, the software components of which it is composed. In order to achieve these goals, the author must achieve the following objectives:

\begin{enumerate}
  \item Describe the context and motivation behind creating an N/CI.
  \item Establish a clear definition as well as the distinctions and advantages of an N/CI.
  \item Identify and define the most relevant aspects required to realise an N/CI in practice.
  \item Illustrate an example architecture of an N/CI to implement its components in practice.
\end{enumerate}

\nomenclature[bci]{BCI}{Brain–computer interface}
\nomenclature[ecog]{ECoG}{Electrocorticography}
\nomenclature[cat]{CAT}{Computed tomography}
\nomenclature[eeg]{EEG}{Electroencephalography}
\nomenclature[bc-i]{B/CI}{Brain/cloud interface}
\nomenclature[nci]{N/CI}{Neural/cloud interface}
