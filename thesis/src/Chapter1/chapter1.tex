\chapter{Introduction}
% \graphicspath{{Chapter1/Figs/}{Chapter1/Figs/}}

\section{First paragraph}
\label{first-paragraph}

Hintergrund und Projektidee darlegen, außerdem klarstellen, warum die Auseinandersetzung mit der Projekt Thematik wichtig, innovativ und relevant ist. Kreative Darstellung der kreativen Idee und Zieldefinition. Wichtig ist hier, dass zwar eine klare Zielvorstellung (Umfang, Pflichtenheft mit Teilzielen, welche Qualität soll erreicht werden) vorhanden ist, aber der genaue Weg noch nicht exakt feststeht. Möglicher Umfang: 10 Prozent der Gesamtlänge.

Goal: Show why your topic is important and attract the reader to your paper. Start with a broad statement and then make it more specific.
- Understanding X is one of the primary objectives of…
- Decades of research have focused on the question…
- The theory that…. is central to…
- It is widely assumed that…
- There has been a long-standing interest in…
- There is general consensus that X is a serious problem…

\section{Middle Paragraph}
\label{middle-paragraph}

Goal: Give an overview of the relevant scientific literature.
What is known? Which questions remain open? Are there conflicts in the literature?
- Several studies have shown that…
- While some studies suggested that X (References) other studies pointed in the opposite
direction (References)
- The findings of some studies suggested that X (References). In contrast, other studies have shown that Y (References)
- Two theories have been proposed to explain this phenomenon. According to theory X….
- Three lines of research are relevant to this question. First,…
- Devin et al. (2003) were one of the first who found evidence that…
- Overall, it has remained unclear whether…
- Taken together, it remains an open question whether…

\section{Last Paragraph(s)}
\label{last-paragraphs}

Goal 1: State your research question
- The goal of the present article is to…
- The research question of the present article is…
Goal 2: State your hypotheses/ predictions
- We hypothesize that…
- We predict that…
- Two hypotheses are conceivable…
- Our primary hypothesis is that…
- Drawing on theory X, we hypothesize that…
Goal 3: Give a rough outline of your research
- We tested whether patients with a diagnosed major depression would report less depressive
feelings after treatment X compared to a placebo treatment.
- To test this hypothesis, we ….
- To answer this question, we …
- For this purpose, we conducted three studies. First,… Second,… Third,…
- To shed more light on this, we used a combination of computer simulations and empirical
studies. First, we used computer simulations to determine what behaviour would arise if theory
3
X is true and what behaviour would arise if theory Y was true. Next, we tested these two
predictions in three empirical studies.
- The present article consists of three sections. In section 1, … In section 2,… In section 3

\section{Goals}
\label{chapter1-goals}

o Does your introduction go from broad (topic) to specific (your research)?
o Did you make clear why your topic is important?
o Did you describe just enough research so that readers can understand how your
research is the next logical step?
o Did you make clear how your research is novel?
o Did you make clear what is speculation, and what are established facts?
o Did you add citations for everything you present as facts?
o Did you use common scientific jargon?
o Did you explain the jargon you use?

\nomenclature[z-IoH]{IoH}{Internet of Humans}
\nomenclature[z-B/CI]{B/CI}{Brain/Cloud Interface}