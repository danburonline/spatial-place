\chapter{Project context}
\graphicspath{{Chapter2/Figs/}{Chapter2/Figs/}}

This chapter describes the project’s context and the current literature findings. The limitations of neuroscience are discussed as well as the state of current non-invasive and sensor-based BCIs, the motivation for developing cloud-based BCI software for the general population, and the broad definition of an N/CI.

\section{Limitations of BCIs}
\label{chapter2-limitations-of-bcis}

The possibilities of BCIs are not without limitations. In addition to hardware limitations, the author addresses a broader issue related to neuropsychology that directly correlates with the software aspects, in addition to the challenges of computability.

\subsection{Decoding neural data}
\label{chapter2-decoding-neural-data}

It is important to emphasise that the task of decoding neural data is different from decoding thoughts, which is a critical factor for BCI software to enable the control and interaction via thoughts. Moreover, decoding neural data and extracting the thoughts behind it so that the software can understand them are disciplines in their own right. This is similar in machine learning use cases: for example, getting computers to recognise letters written on a photograph is a very different problem from interpreting the written words in sentences (i.e. computer vision and natural language processing). Another part is understanding the sentences and their meaning, as in natural language understanding (NLU).

NLU is considered an artificial intelligence (AI) hard problem, which means that the difficulty of these computational problems is assumed to be equivalent to solving the central problem of artificial general intelligence\footnote{Based on the assumption that general human-level intelligence could be computable.} \citep{demasi_theoretical_2010}. Understanding less structured data, such as neural data, is more complex than understanding structured and human-invented syntax such as written language because it contains more hidden features and semantics than a paragraph of text. As a result, the author assumes that completely understanding neural data can also be considered an AI-hard problem. This constraint illustrates well how far current research is from being able to interpret a person’s thoughts based on measured neural data.

\subsection{Abstract thoughts}
\label{chapter2-abstract-thoughts}

To further emphasise the complexity of interpreting neural data, a practical example will be presented: \textit{Imagine a red house in the middle of a forest.} Depending on the individual thought process, one might imagine the house with temporary visual imagery in mind, as in visual thinking, or one might imagine it more verbally, such as conceptually comprehending each word sequentially of what a red house is and that it is located in a forest \citep{amit_asymmetrical_2017}. \autoref{fig:visual-thinking} illustrates the range of options.

\begin{figure}[ht]
  \centering
  \includegraphics[width=\linewidth]{visual-thinking.png}
  \caption{Difference between verbal and visual thinking using the target sentence of a red house in the middle of a forest.}
  \label{fig:visual-thinking}
\end{figure}

Additionally, it should be considered that different types of thoughts exist at different levels of abstraction and complexity. One can assume that the visual image of a red house in the forest is more abstract and far-fetched than, say, the movement of one’s own thumbs, which has a clear physical counterpart. It becomes even more complicated when one imagines abstract concepts that cannot be visualised, such as the idea of a company. A company is an abstract, collectively agreed-upon concept that lacks a clear physical counterpart\footnote{Some people might think of a company building when imagining a company, others might imagine their website, their logo, or physical products.}; it is, therefore, even more complex to decode the meaning of measured brain activity in this case than with the red house.

\subsection{Technological limitations}
\label{chapter2-technological-limitations}

Even though mind reading or decoding abstract thoughts seems difficult, some functional tasks of the brain are still extractable due to their localisation. Localisation means that these signals are generated by local brain areas that can be identified, such as the motor cortex, which has been shown to be responsible for muscle movement (see \autoref{fig:fmri-scan}).

\begin{figure}[ht]
  \centering
  \includegraphics[width=\linewidth]{fmri-scan.jpg}
  \caption[Localised neurons during right- and left- thumb movement using functional magnetic resonance imaging (fMRI).]{Localised neurons during right- and left-thumb movement using functional magnetic resonance imaging (fMRI; \cite{rashid_bilateral_2018}).}
  \label{fig:fmri-scan}
\end{figure}

Examining the areas of the brain responsible for activating individual muscle strands can yield a comparable response of muscle stimulation in the brain and thus be measured as input for BCI software, for example, to move a prosthesis. However, the more specific, behavioural, and abstract the thoughts are, the less spatially visible are the responsible brain areas. By reference to the intention of identifying, for example, the thought of a red house in a forest, the author has identified three technological limitations:

\begin{itemize}
  \item To understand single thoughts, it is essential to have sufficiently clear data with a certain level of detail (e.g., the level of detail that elicits the action potentials of individual neurons\footnote{Action potentials are the fundamental neurobiological and neurochemical processes through which neurons transfer information to each other.}) and temporal precision (an action potential has a short duration of about one millisecond [\cite{byrne_resting_2021}]) to perform studies to extract possible localisation of individual thoughts. Current neuroimaging technologies cannot capture in sufficient detail every process of the entire brain at once to extract the activity of individual neurons while also having high temporal precision.
  \item Even if we could measure every single neuron in the brain with high temporal precision, we would have an extreme amount of data generated concisely. Suppose we would collect a float\footnote{The size of a float on a Windows 64-bit application is 4 bytes which was used for the calculation.} per neuron that represents the rate of change in voltage with respect to time with a frequency of 1 millisecond and then record each neuron in the brain a million times a second (taking into account that the average human brain has around 86 billion neurons); under these conditions, we would generate 305.53337637684 petabytes of data per second. This is currently not feasible for commercially available storage and processing resources.
  \item Even if we had the technology, it would still be challenging because of the difficulty of reproducing experiments in neuroscientific studies—usually referenced as the ‘replication crisis’ \citep{maxwell_is_2015}. It is probably impossible to generate clean-slate brain data that is comparable to previously recorded data since our neurophysiological brain tissue changes over time due to neuroplasticity \citep{nierhaus_immediate_2021}. Moreover, we are in different states of mind every millisecond of our existence, which can produce different effects, such as insufficient sleep, being disturbed by something, mental distraction due to an important event that may have occurred since the last measurement, or a salient thought that randomly occurs while recording neural data.
\end{itemize}

\subsection{Lack of data}
\label{chapter2-lack-of-data}

As mentioned in the previous section, the last two points depend on advances in storage systems or the possibility that we do not actually need such precise brain data to understand single thoughts. However, to address the first point, some promising solutions already exist for measuring large parts of the brain with high temporal and spatial precision, such as time-domain functional near-infrared spectroscopy (TD-fNIRS), which the company Kernel employs in its Flow device \citep{ban_kernel_2021}. TD-fNIRS sensors detect changes in concentrations of oxygenated and deoxygenated brain cell activity by using near-infrared light in response to neuronal activity. According to Kernel, the precision of TD-fNIRS is sufficient for a clearer understanding of the brain and using it for BCI applications. The company, however, claim that collecting and organising longitudinal brain data from a variety of subjects is the key to solving the most difficult challenges in neuroscience \citep{kernel_hello-humanitypdf_nodate}.

Building on Kernel’s claim, a recent publication also claims that even datasets with several hundred people are too tiny to offer consistent insights into the brain; as a result, most published neuroscience studies with dozens or even hundreds of people could all be incorrect in their conclusions \citep{marek_reproducible_2022}. In neuroscientific studies, brain tissue and activity variations have been linked to variances in cognitive capacity, mental health, and other behavioural features which set an important foundation for our current understanding of the brain. Neuromarkers\footnote{A neuromarker is a biomarker that is based on neuroscientific data to detect biological properties such as a disease or illness. For more information, the interested reader can refer to \citeauthor{jollans_neuromarkers_2018} \parencite*{jollans_neuromarkers_2018}.} of behavioural features are frequently sought in such studies to further understand the brain. \citeauthor{marek_reproducible_2022} \parencite*{marek_reproducible_2022} claim that most of the neuromarkers would not work when the collected dataset is more extensive, which would pose a general problem for the field of neuroscience. UK Biobank’s collection of brain scans is one of the first efforts to solve this problem \citep{noauthor_imaging_nodate}, but it is still far from what we might need since, as \citeauthor{marek_reproducible_2022} \parencite*{marek_reproducible_2022} claim, we might even need millions of datasets to start understanding the brain \citep{callaway_can_2022}. This is both fascinating and a possible significant constraint for BCIs, because understanding the brain is essential to making sense of the measured data they interface with.

A counterargument, however, is offered by Andrew Ng, artificial intelligence (AI) pioneer and founder of the Google Brain research lab. Ng believes that machine learning, which underpins all BCI software, should be developed in a data-centric manner, which means that quality should be preferred over quantity. That is, the quality of the data on which models are trained should be as high as possible to answer specific research questions rather than focusing on merely collecting a huge amount of data \citep{brown_why_2022}. However, this brings one back to the replication crisis, which is the difficulty in generating clean or universally valid brain data comparable to previously collected data due to the nature of our ever-changing brain.

Ultimately, there will almost certainly always be a mix of both approaches. As a result, for generally applicable and mass-market-ready BCIs, a relatively large amount of qualitative brain data collected in specific and reproducible experiments or environments is required. This is where high-end, customer-focused BCIs could come into play because the adoption rate of a device suitable for everyday use is higher than the number of subjects in research labs, resulting in larger and more longitudinal datasets. This, combined with more targeted experiments, improved neuroimaging technologies, and advances in machine learning, could unlock enormous potential in brain research.

\section{BCI landscape}
\label{chapter2-research-landscape}

This section will discuss the current landscape of customer-focused and non-invasive BCIs, their applications, and the distinctions within their software offerings.

\subsection{Real-world BCI applications}
\label{chapter2-real-world-bci-applications}

As mentioned in \autoref{chapter1-background}, consumer-focused BCI products are already commercially available. OpenBCI, a non-medical BCI company, does not provide a specific use case, but they provide hardware (as depicted on \autoref{fig:openbci}) as well as software that is universally applicable. These can be used in research wherever EEG is employed and in developing BCI applications. Several neurofeedback or research apps have been created using OpenBCI’s products \citep{openbci_openbci_nodate}. Taking this information into consideration, one can see that the OpenBCI customer is responsible for developing their own BCI applications or incorporating it into their research, rather than having a sophisticated end-user application directly from OpenBCI.

\begin{figure}[!ht]
  \minipage{0.32\textwidth}
  \includegraphics[width=\linewidth]{openbci.jpeg}
  \caption[OpenBCI’s EEG device.]{OpenBCI’s EEG \\ device \citep{be_superhvman_conor_2017}.}
  \label{fig:openbci}
  \endminipage\hfill
  \minipage{0.32\textwidth}
  \includegraphics[width=\linewidth]{nextmind.jpeg}
  \caption[NextMind’s BCI device.]{NextMind’s BCI \\ device \citep{louise_neurotechnology_2019}.}
  \label{fig:nextmind}
  \endminipage\hfill
  \minipage{0.32\textwidth}%
  \includegraphics[width=\linewidth]{muse.jpeg}
  \caption[Muse’s meditation headband.]{Muse’s meditation \\ headband \citep{muse_muse_nodate}.}
  \label{fig:muse}
  \endminipage
\end{figure}

Another example of a commercial BCI is NextMind’s product, as shown in \autoref{fig:nextmind}. The company does not focus on having an end-user application for its BCI, but focuses instead on offering a software development kit (SDK) so that the Unity real-time engine can use NextMind’s technology for brain-controlled actions in video games. One significant difference between NextMind and OpenBCI is that NextMind includes a built-in classification of neural data captured by hardware—in this case, classification of active visual focus on virtual objects based on steady-state visual evoked potentials (SSVEP). Because its business model is presumably based on the unique selling proposition of its active visual focus classifier, NextMind does not provide access to the raw EEG data collected by the sensors. Nonetheless, NextMind’s product is less focused on a specific use case, as it is applicable to all kind of games inside the Unity engine. A relatively closed and specific BCI is illustrated by the EEG headband by Muse, as shown in \autoref{fig:muse}. Its purpose is to measure meditation and sleep. The company also offers an end-user app to help people better understand their meditation and sleep and how to improve them. Compared to the previously mentioned products, the Muse headband is not a unidirectional BCI per se as there is also biofeedback based on neural data. However, the key difference between Muse and OpenBCI is that the neurotechnology has been abstracted. Users do not need to know anything about neuroscience, neurotechnology, or the interpretation and classification of neural data to achieve useful functionality for their use case. They also do not need to understand the software system’s underlying architecture. They only need to know how to pair the device with their smartphone via Bluetooth.

Aside from full-stack BCI solutions, in which a company provides a complete BCI solution, including hardware and software, some companies focus solely on the software aspect. One such example is Neuromore, a company that provides a neural signal processing software platform. The company is hardware agnostic, which means one can plug nearly any BCI hardware or sensor into their computer and connect it to their Neuromore Studio software.

\begin{figure}[ht]
  \centering
  \includegraphics[width=\linewidth]{neuromore.png}
  \caption[Screenshot of the Neuromore Studio software.]{Screenshot of the Neuromore Studio software \citep{neuromore_neuromore_nodate}.}
  \label{fig:neuromore}
\end{figure}

Neuromore Studio, as shown in \autoref{fig:neuromore}, is free and open-source software that runs locally on various platforms. It provides a variety of drag-and-drop interfaces for creating and managing signal processing pipelines. For example, one can transform EEG data to extract band power, create triggers based on band-power selection, and generate conditional outputs to perform tasks such as moving a character in a video game. The author aims to differentiate the offerings of these consumer-oriented BCIs along a spectrum. On one side of the spectrum are BCI companies that provide the hardware (with software that at least connects to the device) and are then more generally applicable to use cases not defined by the company behind the BCI (such as OpenBCI). On the other side are BCI companies that are application-specific in terms of both the software and the hardware, such as the Muse headband.

Although this thesis focuses on consumer-oriented BCIs, the applications of various BCI offerings can still be distinguished based on whether they are more consumer-oriented or research-oriented—such as the distinction, for example, between NeuroSky (a company creating EEG-based BCIs for hobbyists) and Emotiv (a company creating professional and expensive EEG systems), which are more research-oriented. However, both NeuroSky and Emotiv provide a research version and a consumer or enterprise version of their software and hardware, aiming for general-purpose applicability across customer segments and use cases. Other considerations include whether the applications are steady-state evoked, such as those based on a frequency of noise laid on top of virtual objects to detect which object the person is looking at (e.g. NextMind), or whether they track the totality of mental states without evoking neural signals with external stimuli, such as in tracking sleep or concentration levels, both of which arise primarily inside the brain. This distinction can be labelled as passive, active, or reactive BCI, as \citeauthor{alimardani_passive_2020} coined in their work on passive BCIs \citep{alimardani_passive_2020}. However, the author does not want to include this dimension because it would introduce additional complexities related to the BCI software application layer.

\begin{figure}[!ht]
  \centering
  \includegraphics[width=\linewidth]{bci-components.png}
  \caption{Architectural overview of BCI components.}
  \label{fig:bci-components}
\end{figure}

The application layer, as shown in \autoref{fig:bci-components}, is the part of a BCI that acquires the interpreted data from a classification model and turns it into applicable functionality to interface with a physical or digital counterpart to perform functions such as moving a player in a game or initiating sound on the computer via its speakers. There is also the physical part, the brain, and a possible physical interaction counterpart in the form of, for example, a robot arm. The totality of the software stack is responsible for processing the data, that is, extracting the relevant information from the raw data and turning it into any desired and meaningful output for the application layer.

\subsection{Unobtrusive hardware and software}
\label{chapter2-unobtrusive-hardware-and-software}

The unobtrusiveness of hardware and software is another aspect to consider when discussing BCIs. Unobtrusiveness in hardware means that it is either not visible at all\footnote{Other things related to the overall user experience (UX) are sometimes included in the notion of unobtrusiveness, such as comfort, reusability, and convenience; however, the author is implying only physical characteristics such as shape and size.}, as when sensors are implanted beneath the skull, or that it is in a form factor that is already socially established. The hardware prototype of IDUN Technologies, a Swiss startup, as shown in \autoref{fig:unobstrusive-hardware}, measures brain activity in the ear canal that aims to resemble the form factor of established in-earbuds. \autoref{fig:obstrusive-hardware} shows the Notion product from Neurosity, which measures EEG on the head and is not comparable to a socially established form factor such as earbuds. What is considered socially established and accepted truly depends on the society and context, as one could argue that wearing a Neurosity device under a hat while talking to a friend is more acceptable than wearing in-earbuds. Still, the implications of different form factors must also be considered, such as the possibility of moving the device and thus creating motion artefacts in the signals or the position of the sensors. The ear canal is ideally located close to the brain’s auditory cortex but not so much to the visual cortex, which is located at the back of the head. However, further hardware implications for BCIs are not a topic covered in this thesis.

\begin{figure}[!ht]
  \minipage{0.49\textwidth}
  \includegraphics[width=\linewidth]{unobtrusive.jpg}
  \caption{IDUN Guardian hardware, \\ rather unobtrusive BCI.}
  \label{fig:unobstrusive-hardware}
  \endminipage\hfill
  \minipage{0.49\textwidth}
  \includegraphics[width=\linewidth]{obtrusive.jpg}
  \caption{Neurosity Notion hardware, \\ rather obtrusive BCI.}
  \label{fig:obstrusive-hardware}
  \endminipage\hfill
\end{figure}

Nonetheless, it is perhaps not as simple to discuss the unobtrusiveness of software as it is with hardware. Unobtrusive\footnote{Other words for unobtrusive could be discreet, fully-integrated, invisible or simply ‘in the background’.} software, as defined by the author, is the abstraction of the underlying software or system that executes the logic to fulfil a task without the user knowing what the technical requirements are. For example, to use an HP ENVY Photo 6200 printer with one’s Android phone, one must first download the HP Smart app and the HP Print Service Plugin app that acts as a driver for the printer to get it set up and running \citep{hp_hp_nodate}. In the case of the HP printer, the user must understand some of the underlying technical requirements in order for it to work, rather than simply concentrating on the task of printing something. An example of unobtrusive software is a computer mouse that one simply plugs in and starts using immediately\footnote{Unobtrusiveness usually correlates with usability, but it is not always the case; more advanced users would not consider locked-in abstraction as more usable.}.

Unobtrusive software in BCI refers to the ability to connect one’s hardware to the computer or smartphone and use it without the need for additional software such as drivers or command-line interface (CLI) software. For example, to use an OpenBCI device, one needs to open the graphical user interface (GUI) app, connect the hardware, presumably test its quality, begin a data stream session, and output the stream via a network system such as a Lab Streaming Layer (LSL), connect to the signal from software such as Neuromore Studio \citep{openbci_neuromore_nodate}, run the data through a classification pipeline, and then connect the output from Neuromore to a video game via the engine itself to have controls for the video game. Clearly, this software is not unobtrusive. There are examples of software included as an executable file, and which is thus relatively unobtrusive; but this software is closely linked with the hardware and the brand behind the hardware, or it is in the proof of concept (PoC) stage rather than a production-grade application. Buying a new pair of headphones and plugging them into one’s computer to enjoy neuro-enhanced\footnote{Neuro-enhanced software can be described as adding additional features to input methods via the brain.} experiences interacting with the brain’s outputs or measuring brain data across all apps and the operating system would be examples of truly unobtrusive BCI software.

\subsection{Production-grade software}
\label{chapter2-production-grade-software}

Another aspect of BCIs is the state they are in, such as the production maturity of the software. There is no clear definition of what production-grade software is; but in most cases, software developers agree on the following characteristics:

\begin{itemize}
  \item Software that works at any time when access is required. It is therefore capable of frequent and intensive use in commercial or industrial environments.
  \item Software whose behaviour is deterministic and predictable and is, therefore, well-tested, well-documented, and optimised in terms of speed, efficiency, and security for the given context (e.g. the size of the user base). Usually, developers agree on a Definition of Done (DoD) inside their team to what is considered production-ready; some examples include test coverage of 80+\%, peer-reviewed and commented code, and a common code style guide.
  \item Software that runs in a production environment—that is, on a cloud computing cluster for actual users rather than in a test environment for test users or on hardware delivered to real customers—and that can adapt itself to the context, such as to a higher access rate or possibly insecure user-generated input. In most cases, especially in cloud computing, production-grade also means larger datasets (such as in databases), the possibility of a greater number of edge cases due to a larger user base, and most importantly, more available computing power on production instances.
\end{itemize}

As stated earlier, most BCIs, such as the OpenBCI, are usually not intended for production. They are intended for PoCs such as controlling objects in games or conducting research. End-to-end and full-stack BCIs for production are rare, as most are highly specific and not intended for general applicability (such as the Muse headband) or the software aspect is intended for PoCs or research (such as with Emotiv). Pure software products, such as the one from Neuromore, lack the hardware component and lack an SDK that can be integrated into existing software for a variety of platforms. Neurosity aims to provide a universally usable and unobtrusive software stack that is even open source. However, because the hardware is not unobtrusive enough, it does not meet the author’s definition as mass-market-ready and production-grade—apart from the fact that it is not known whether their software stack is aimed to be used in production \citep{neurosity_neurosity_2022} due to third-party developers providing disclaimers that it is a work in progress \citep{turney_notion_2022}). Companies such as Neuralink are presumably working on a general-purpose, unobtrusive, and production-grade software system that enables developers to build production apps and even platforms on top of it for a variety of use cases without being limited \citep{musk_integrated_2019}. Since the intended hardware is also unobtrusive in its form factor (being implanted), Neuralink has a high potential to become one of the first general applicable and mass-market-ready BCIs if one ignores that surgery is required to acquire the device \citep{neuralink_approach_nodate}.

\section{Definition of an N/CI}
\label{chapter2-definition-of-an-nci}

All the previously mentioned aspects feed into the definition and motivation of an N/CI, which the author introduced as a new term in \autoref{chapter1-supposition}. \autoref{appendix1-definition-and-motivation-of-an-nci} goes into more detail about the motivation and clear definition of an N/CI, and the aspects of cloud computing in combination with BCI software that can be displayed in a three-dimensional axis, as shown in \autoref{fig:nci-definition-intro}. The invention of the term ‘N/CI’ was a byproduct of this project’s implementation and will be discussed in more detail in \autoref{chapter4-invention-of-nci}.

\begin{figure}[ht]
  \centering
  \includegraphics[width=\linewidth]{thesis/src/Appendix1/Figs/nci-definition.png}
  \caption[Visualisation of the term neural/cloud interface with its three axes and differentiation of six terms.]{Visualisation of the term neural/cloud interface with its three axes and differentiation of six terms.}
  \label{fig:nci-definition-intro}
\end{figure}

\nomenclature[nlu]{NLU}{Natural language understanding}
\nomenclature[ai]{AI}{Artificial intelligence}
\nomenclature[fmri]{fMRI}{Functional magnetic resonance imaging}
\nomenclature[tdfnirs]{TD-fNIRS}{Time-domain functional near-infrared spectroscopy}
\nomenclature[ssvep]{SSVEP}{Steady-state visual evoked potential}
\nomenclature[ux]{UX}{User experience}
\nomenclature[cli]{CLI}{Command line interface}
\nomenclature[gui]{GUI}{Graphical user interface}
\nomenclature[poc]{PoC}{Proof of concept}
\nomenclature[lsl]{LSL}{Lab streaming layer}
\nomenclature[it]{IT}{Information technology}
\nomenclature[gpu]{GPU}{Graphics processing units}
\nomenclature[cpu]{CPU}{Central processing units}
\nomenclature[iot]{IoT}{Internet of Things}
\nomenclature[api]{API}{Application programming interface}
\nomenclature[sdk]{SDK}{Software development kit}
