\chapter{Methodologies}
\graphicspath{{Chapter3/Figs/}{Chapter3/Figs/}}

This chapter describes the project-related academic methodologies in the given context of the author and the planned approach to achieve the thesis's goals and objectives. The reader is introduced to the rationale for the intended workflows, hardware, and software tools.

\section{Derivation of the case study}
\label{chapter3-derivation-of-the-case-study}

In October 2021, the author began working as a cloud software engineer at IDUN Technologies, an ETH spin-off startup in Zürich, to further develop their existing software products. IDUN had already created a PoC software system that included a web-based single-page application (SPA) hosted on AWS Amplify, a backend-as-a-service product aiming to simplify the deployment of backends for mobile and web apps. IDUN's in-ear sensor sent EEG data to a physical network bridge via Bluetooth and then to the cloud via the internet. The raw EEG data was saved and available for download in various file formats.

\begin{figure}[!ht]
  \centering
  \includegraphics[width=\linewidth]{raw-filtered-data.png}
  \caption{Difference between raw and filtered EEG data from IDUN's in-ear device.}
  \label{fig:raw-filtered-eeg}
\end{figure}

In addition to raw data, the IDUN SPA provided transformed data, such as, e.g. filtered data, which included low-pass and high-pass filtering of EEG data as shown on \autoref{fig:raw-filtered-eeg}. This transformed EEG data was then saved alongside the raw version on the cloud.

The EEG data could also be visualised in near-real-time on the SPA as a time-series x- and y-axis plot. Additionally, users could control the device by sending start and stop commands to the hardware components. The architecture overview is displayed on \autoref{fig:idun-amplify} in which IDUN GDK refers to the IDUN Guardian Development Kit.

\begin{figure}[!ht]
  \centering
  \includegraphics[width=\linewidth]{idun-amplify.png}
  \caption{IDUN's software architecture at the end of 2021.}
  \label{fig:idun-amplify}
\end{figure}

The system was in a relatively unstable state and had strange error sources that made it impossible to reliably record EEG data for more than a few minutes, making the product unusable for existing customers or a mass-market launch. While working on the system, the author encountered various technological bugs and flaws as described in the following list:

\begin{itemize}

\item AWS Amplify is great for frontend developers who want to build simple backends with CRUD\footnote{CRUD is an acronym describing general operations of a backend system: Create, Read, Update and Delete.} operations, but it is not intended for anything custom-made, such as, e.g. the streaming-focused aspect of EEG data. Therefore, AWS Amplify must be abandoned as soon as possible, or the project will be built with the wrong tools and foundations.
\item The network bridge was a Raspberry Pi 4 Model B running Python code, which after some analysis, turned out to be the primary source of most of the bugs due to limits in computational power. The question was raised whether the network bridge was needed since IDUN intends to build a completely mobile system, not tied to a network and directly connected to, e.g. a mobile phone.
\item The cloud's heartbeat functionality was missing, which meant that the cloud knew nothing about the hardware devices, and simply assumed that data would flow in as soon as the start command was sent to the device, creating a pure happy path scenario.
\item The cloud infrastructure was automatically provisioned by AWS Amplify, which uses AWS CloudFormation as its core, which is an infrastructure as code (IaC) tool, that ran inside of AWS CodePipeline, AWS' continuous integration service, which made everything coupled to specific AWS services where the technical decision to use them was not made based on reasoning but based on Amplify's creators to utilise them.
\item All software in the cloud was built using the AWS Console (the AWS GUI). Therefore, no current state in the form of IaC reflected the current infrastructure, which made it impossible to reproduce the cloud in different environments, e.g. prevent blast radius' if something went wrong.
\item The data was streamed via IBM messaging and queuing middleware telemetry transport (MQTT), a publish-and-subscribe transport protocol commonly used for IoT devices that, e.g. regularly send telemetry data. The purpose of MQTT was not to send high-frequency EEG data in real-time but rather to minimise network bandwidth. Therefore, there was also a need to rethink this technological decision based on the nature of IDUN's EEG sensor, which produces 250 EEG samples per second.
\item The SPA was a thick client, meaning that it ran a lot of business logic, such as filtering raw data for real-time visualisation. This was another technological misstep, as the client-side JavaScript ecosystem is far inferior compared to, e.g. the Python ecosystem that could run in the backend to handle such tasks. If possible, shipping business logic that could hold intellectual property to clients should be avoided if possible, especially with a commercial product.
\item The SPA was not connected to a single endpoint on the backend and used the MQTT stream and the non-real-time aspects of the app (e.g. login or list of recorded EEG data) via different sources. For example, the MQTT stream was subscribed directly from the device itself and did not run through AWS Amplify's API, which made coupling the systems into a coherent and robust API cumbersome.
\item The state of the whole system was difficult to handle due to the decoupled logic from the MQTT stream and Amplify's API. Combined with the lack of a hardware heartbeat, it was very tedious to figure out what the user was doing and what was being sent. As an interim solution, AWS ElastiCache, a provisioned service for in-memory databases like Redis, was set up. Accordingly the state was both handled in the SPA and on ElastiCache simultaneously which enacted even new problems such as sending EEG data to the void if the user closed the browser during a data stream.
\end{itemize}

Due to growing problems with the existing software system and an ever-increasing technical debt as a result of the software not being test-driven or developed without clean code quality standards, resulting in bugs and quirks that are difficult to track down, the author proposed to halt the implementation of new features and restructure the system from the ground up using a more software engineering-oriented approach. The company's management approved the request for this redevelopment in early December 2021. At the time, the author was already working on his original Bachelor project, which focused on a mind-controlled multiplayer game, assuming that IDUN's software system would be stable by the time the Bachelor project began. The original Bachelor project's focus was officially changed at the end of 2021 to create a thesis on the redevelopment of IDUN's cloud software.

\section{Case study}
\label{chapter3-case-study}

As mentioned in previous chapters, IDUN manufactures an EEG sensor in the form of in-ear headphones. Their vision is to sell their hardware and license a software product coupled with the hardware called Neuro-Intelligence Platform (NIP), as shown on \autoref{fig:idun-nip}.

\begin{figure}[!ht]
  \centering
  \includegraphics[width=0.75\linewidth]{idun-nip.png}
  \caption{IDUN's vision of a closed neurofeedback loop \citep{idun_guardian_nodate}.}
  \label{fig:idun-nip}
\end{figure}

The mission is essential that an unobtrusive BCI device, such as an in-ear headset, can be worn for the majority of the day while measuring neural data that is sent in real-time to the cloud to process and classify actionable insights that other developers can use to build their interventions (e.g., apps, websites, and games) on top of it to promote mental health and well-being. Furthermore, the aspect of longitudinal data is essential, implying that it could be advantageous to store neural data over a long period and run classifiers on it from time to time to understand one's brain better.

It is essential to select an appropriate research method to develop a system that would fulfil the company's mission and vision. The author chose a case study because it is an effective method for dealing with unusual and atypical cases while providing new and unexpected perspectives in certain situations such as in a deep-tech startup like IDUN.

\section{Procedure}
\label{chapter3-procedure}

This section describes the planned procedure for conducting a case study-based research methodology to develop the proposed N/CI at IDUN Technologies as part of their NIP.

\subsection{Project stages}
\label{chapter3-project-stages}

The goal was to conduct qualitative research and examine the use case from various perspectives. In addition, two other methodologies were used in the case study: 1. Expert interviews, which entails locating experts on topics such as cloud or BCI and asking them questions that will assist in answering implementation questions, and 2. group discussions, which aim at learning about people's attitudes and opinions about related topics.

\begin{figure}[!ht]
  \centering
  \includegraphics[width=0.75\linewidth]{idun-timeline.png}
  \caption{IDUN's plan to achieve their goals \citep{idun_guardian_nodate}.}
  \label{fig:idun-timeline}
\end{figure}

Before defining the project stages, it is essential to look at the timing of IDUN's roadmap as shown on \autoref{fig:idun-timeline}. There are three essential pieces of information: The stage in which the technology resides, the estimated user base size, and the time frames of these stages that want to be achieved. The author was working towards a version starting to be released in 2023, ergo working on the N/CI aimed at the Guardian deployment phase. As the company is still implementing the first neuromarkers into the product and is focusing on four use cases due to the size of the team and the mass production of hardware and sensors, the sales team is targeting the sale of devices to about 100'000 people.

Another constraint is the deadline of the bachelor's thesis of the author, which is set on the 5th of August 2022. User size, company stages and the deadline for the thesis are important context information for defining the project stages for this use case. Following on \autoref{tab:project-stages} are the proposed project stages to execute the case study:

\begin{table}[!ht]
\centering
\resizebox{\textwidth}{!}{%
\begin{tabular}{
>{\columncolor[HTML]{FFFFFF}}l l}
\cellcolor[HTML]{000000}{\color[HTML]{FFFFFF} Project stage} &
  \cellcolor[HTML]{000000}{\color[HTML]{FFFFFF} Description} \\ \hline
\multicolumn{1}{|l|}{\cellcolor[HTML]{FFFFFF}\textbf{\begin{tabular}[c]{@{}l@{}}1.1 Define key technical\\ requirements and\\ constraints\end{tabular}}} &
  \multicolumn{1}{l|}{\begin{tabular}[c]{@{}l@{}}Based on the internal hardware, the neuroscience and data science departments and the IDUN mission\\ and roadmap, the author needs to define the purely technical requirements and constraints. Some\\ constraints are e.g. hiring of advisors, time limits and risks due to investment rounds. Some purely\\ technical requirements come from e.g. the firmware team or the material scientists, such as maximum\\ latency or the data structure of the digitalised EEG.\end{tabular}} \\ \hline
\multicolumn{1}{|l|}{\cellcolor[HTML]{FFFFFF}\textbf{\begin{tabular}[c]{@{}l@{}}1.2 Extensive literature\\ research\end{tabular}}} &
  \multicolumn{1}{l|}{\begin{tabular}[c]{@{}l@{}}Already since the beginning of the original bachelor thesis, the author started to do literature research in\\ the field of BCI, which is very helpful for the new focus. Further literature review now includes books\\ on data intensive applications and cloud, articles and research papers.\end{tabular}} \\ \hline
\multicolumn{1}{|l|}{\cellcolor[HTML]{FFFFFF}\textbf{\begin{tabular}[c]{@{}l@{}}2.1 External user\\ interviews\end{tabular}}} &
  \multicolumn{1}{l|}{\begin{tabular}[c]{@{}l@{}}Defining users personas with IDUN's product manager, application engineer and sales team, finding\\ real people to represent these personas, preparing an external interview framework and questions,\\ gathering as many insights as possible.\end{tabular}} \\ \hline
\multicolumn{1}{|l|}{\cellcolor[HTML]{FFFFFF}\textbf{\begin{tabular}[c]{@{}l@{}}2.2 Creative workshop and\\ prototyping\end{tabular}}} &
  \multicolumn{1}{l|}{\begin{tabular}[c]{@{}l@{}}Based on the insights from the user interviews, conduct a creative workshop with IDUN's product\\ manager and application engineer. The results are design artefacts such as wireframes, user flows,\\ interactive prototypes and architecture diagrams.\end{tabular}} \\ \hline
\multicolumn{1}{|l|}{\cellcolor[HTML]{FFFFFF}\textbf{\begin{tabular}[c]{@{}l@{}}3.1 Internal group\\ discussions\end{tabular}}} &
  \multicolumn{1}{l|}{\begin{tabular}[c]{@{}l@{}}The design artefacts are used for an internal validation with the department heads. Several group\\ discussions are held with each department, ranging from materials science to business development.\\ Based on the group discussions, the artefacts are adapted and improved.\end{tabular}} \\ \hline
\multicolumn{1}{|l|}{\cellcolor[HTML]{FFFFFF}\textbf{3.2 Expert interviews}} &
  \multicolumn{1}{l|}{\begin{tabular}[c]{@{}l@{}}Define experts in different areas of BCI, cloud and EEG, and neuroethics. Create a framework for\\ expert interviews and questions that are still unclear or open based on the design artefacts. Conduct\\ expert interviews and gather as many insights as possible to adapt and improve the design artefacts.\end{tabular}} \\ \hline
\multicolumn{1}{|l|}{\cellcolor[HTML]{FFFFFF}\textbf{4. Start bootstrapping}} &
  \multicolumn{1}{l|}{\begin{tabular}[c]{@{}l@{}}While the design process is still ongoing, you can already start implementing the most important\\ technical requirements, with anything that is a flexible bootstrapping of internal development\\ processes to ensure quality assurance, for example, not being tied to the other phases.\end{tabular}} \\ \hline
\multicolumn{1}{|l|}{\cellcolor[HTML]{FFFFFF}\textbf{\begin{tabular}[c]{@{}l@{}}5. Iterative and agile\\ implementation via\\ Scrum\end{tabular}}} &
  \multicolumn{1}{l|}{\begin{tabular}[c]{@{}l@{}}IDUN works with Scrum, for this the author has introduced Scrum in the research and development\\ department. Scrum at IDUN has three-week sprints and there are ten sprints from January 2022 to the\\ end of July 2022 that can be used to go through the design process and implement the designs. In the\\ process, there are several iterative processes to validate and test the increments of the system based on\\ insights from the implementation, ongoing literature research, insights from other departments or the\\ design process that is still ongoing.\end{tabular}} \\ \hline
\end{tabular}%
}
\vspace{10pt}
\caption{Project stages to answer the research question of this thesis.}
\vspace{-5pt}
\label{tab:project-stages}
\end{table}

Number 1.1 is crucial before starting anything else. Number 1.2 is continuously in progress. Numbers 2.1 to 3.2 take place simultaneously while number 4 is ongoing\footnote{Number 4 refers to bootstrapping, which usually refers to the practice of setting up development environments that include, e.g. initial Git repositories, folder structures and creating accounts on cloud providers.}. As soon as numbers 3.2 and 4 are completed, number 5 can be started.

\subsection{Group discussions}
\label{chapter3-group-discussions}

According to the author, it is critical to conduct user interviews with external people who do not have too much insider information about IDUN to avoid developing a very technical product that only someone with much company-specific knowledge can use. Based on the user interviews, the goal is to create design artefacts in Figma for everything GUI-related and in Miro for everything architecture-related. These design artefacts are then used to communicate the creative design process to IDUN's internal staff based on the findings from the external user interviews. The ideas and findings should be presented and validated concerning IDUN staff's understanding and product perceptions. Because the development of a complete neurotechnology product involves many disciplines ranging from physics to machine learning, the company must have a shared understanding of what the software component of the NIP, ergo the N/CI, will look like and what for functionalities it will have.

\subsection{Expert interviews}
\label{chapter3-expert-interviews}

It is critical to consult experts after the external and internal validation and design process and its iteration through group discussions. In the author's experience, it is difficult to, e.g. engage consultants at an early stage of the process without knowing what one wants to build because advisors, e.g. from a consultancy firm, tend to push for commercially oriented ideas to generate their revenue rather than thinking about the project's long-term success. Experts are chosen based on the author's experience and the experience of other IDUN Technologies employees, such as data scientists and neuroscientists. According to a previous blog post by the author, the importance of ethics in the field of BCI is critical \citep{burger_influence_2022}. As a result, it is crucial to include neuroethics experts in the plans and ideas for IDUN's N/CI.

\section{Software and tools}
\label{chapter3-software-and-tools}

There is a great amount of software and tools that can be used to construct a comprehensive software system, such as IDUN's N/CI. The author had some leeway in deciding which software and tools (i.e. libraries and frameworks) to use. Nonetheless, some software and tool limitations should be considered before beginning the implementation:

\begin{itemize}
    \item IDUN has received a large number of credits from Amazon for its AWS account, which should therefore be used to avoid creating redundant costs when using another cloud provider.
    \item The Python ecosystem has strong roots among engineers and researchers at IDUN, which should be taken into account too. An example of this is Python's Jupyter Notebooks, a state-of-the-art and software tool used at IDUN and in the data science industry.
    \item The author himself needs to be comfortable with the existing knowledge of software and tools for building IDUN's N/CI; otherwise, he would have to relearn everything before the actual implementation, which would make things even more complicated. The AWS cloud, for example, would be one such case where the author initially had more knowledge and experience with the Google Cloud Platform, so the need to learn an entire cloud provider is in front of him.
    \item Any new tool that is not open source and free to use should be carefully considered in terms of future maintainability, security and learning effort.
    \item Wherever possible, redundancy in the workload should be avoided, as a venture capital-funded startup like IDUN has a particular runway time that describes the company's remaining time before it runs out of cash. Spending, e.g. an entire sprint working on something that is then thrown away, can be business-critical, so every software and tool decision is carefully weighed.
\end{itemize}

\section{Checkpoints for progress}
\label{chapter3-checkpoints-for-progress}

Progress is reviewed at the end of each three-week sprint. IDUN's entire research and development team and stakeholders, such as C-level management, attend the review events. Each department asks questions about the author's progress and findings. The planning for the upcoming sprint will take place based on the topics discussed in the reviews. Asana, a project management tool, mirrors the overall product backlog and the personal backlog of the author's bachelor project and tracks progress in terms of story points and remaining epics in its dashboard. In addition to Asana, documentation of all project phases, as mentioned in \autoref{tab:project-stages}, will be documented in Notion. Screenshots of the mentioned artefacts will be attached in the appendices.

The author wants to emphasise an important point: the quality of the work is an essential component of the chosen methodology, made even more crucial when one considers the extensive list of flaws in the current IDUN software system. This kind of thing can only be avoided by using extensive quality assurance measures (i.e. unit tests and continuous integration) during the bootstrapping project stage. To quote Robert C. Martin from his Clean Architecture book: “The only way to go fast, is to go well” \citep{martin_clean_2018}.

\nomenclature[nip]{NIP}{Neuro-Intelligence Platform}
\nomenclature[spa]{SPA}{Single-page application}
\nomenclature[gdk]{GDK}{Guardian Development Kit}
\nomenclature[mqtt]{MQTT}{IBM messaging and queuing middleware telemetry transport}
